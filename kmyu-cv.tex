%%%%%%%%%%%%%%%%%%%%%%%%%%%%%%%%%%%%%%%%%%%%%%%%%%%%%%%%%%%%%%%%%%%%%%%%
%%%%%%%%%%%%%%%%%%%%%% Simple LaTeX CV Template %%%%%%%%%%%%%%%%%%%%%%%%
%%%%%%%%%%%%%%%%%%%%%%%%%%%%%%%%%%%%%%%%%%%%%%%%%%%%%%%%%%%%%%%%%%%%%%%%

%%%%%%%%%%%%%%%%%%%%%%%%%%%%%%%%%%%%%%%%%%%%%%%%%%%%%%%%%%%%%%%%%%%%%%%%
%% NOTE: If you find that it says                                     %%
%%                                                                    %%
%%                           1 of ??                                  %%
%%                                                                    %%
%% at the bottom of your first page, this means that the AUX file     %%
%% was not available when you ran LaTeX on this source. Simply RERUN  %%
%% LaTeX to get the ``??'' replaced with the number of the last page  %%
%% of the document. The AUX file will be generated on the first run   %%
%% of LaTeX and used on the second run to fill in all of the          %%
%% references.                                                        %%
%%%%%%%%%%%%%%%%%%%%%%%%%%%%%%%%%%%%%%%%%%%%%%%%%%%%%%%%%%%%%%%%%%%%%%%%

%%%%%%%%%%%%%%%%%%%%%%%%%%%% Document Setup %%%%%%%%%%%%%%%%%%%%%%%%%%%%

% Don't like 10pt? Try 11pt or 12pt
\documentclass[10pt]{article}
\RequirePackage[T1]{fontenc}

% LaTeX will typeset using Computer Modern Roman, which a lot of
% non-mathematicians and non-engineers won't like. Also, a few PDF
% viewers may not render CMR very well. Instead, Times New Roman can
% be used. That's what this package does.
\usepackage{times}

% The automated optical recognition software used to digitize resume
% information works best with fonts that do not have serifs. This
% command uses a sans serif font throughout. Uncomment both lines (or at
% least the second) to restore a Roman font (i.e., a font with serifs).
% (NOTE: This requires the times package above)
%\renewcommand{\familydefault}{\sfdefault}

% This is a helpful package that puts math inside length specifications
\usepackage{calc}

% This package helps LaTeX auto-hyphenate hyphenated words if you use
% special hyphens. For example, bio\-/mimicry will properly hyphenate
% ``mimicry'' if necessary.
\usepackage[shortcuts]{extdash}

% Layout: Puts the section titles on left side of page
\reversemarginpar

%
%         PAPER SIZE, PAGE NUMBER, AND DOCUMENT LAYOUT NOTES:
%
% The next \usepackage line changes the layout for CV style section
% headings as marginal notes. It also sets up the paper size as either
% letter or A4. By default, letter was used. If A4 paper is desired,
% comment out the letterpaper lines and uncomment the a4paper lines.
%
% As you can see, the margin widths and section title widths can be
% easily adjusted.
%
% ALSO: Notice that the includefoot option can be commented OUT in order
% to put the PAGE NUMBER *IN* the bottom margin. This will make the
% effective text area larger.
%
% IF YOU WISH TO REMOVE THE ``of LASTPAGE'' next to each page number,
% see the note about the +LP and -LP lines below. Comment out the +LP
% and uncomment the -LP.
%
% IF YOU WISH TO REMOVE PAGE NUMBERS, be sure that the includefoot line
% is uncommented and ALSO uncomment the \pagestyle{empty} a few lines
% below.
%

%% Use these lines for letter-sized paper
\usepackage[paper=letterpaper,
            %includefoot, % Uncomment to put page number above margin
            marginparwidth=1.2in,     % Length of section titles
            marginparsep=.05in,       % Space between titles and text
            margin=1in,               % 1 inch margins
            includemp]{geometry}

%% Use these lines for A4-sized paper
%\usepackage[paper=a4paper,
%            %includefoot, % Uncomment to put page number above margin
%            marginparwidth=30.5mm,    % Length of section titles
%            marginparsep=1.5mm,       % Space between titles and text
%            margin=25mm,              % 25mm margins
%            includemp]{geometry}

%% More layout: Get rid of indenting throughout entire document
\setlength{\parindent}{0in}

% Provides special list environments and macros to create new ones
\usepackage[shortlabels]{enumitem}

% Simpler bibsections for CV sections
% (thanks to natbib for inspiration)
%
% * For lists of references with hanging indents and no numbers:
%
%   \begin{bibsection}
%       \item ...
%   \end{bibsection}
%
% * For numbered lists of references (with hanging indents):
%
%   \begin{bibenum}
%       \item ...
%   \end{bibenum}
%
%   Note that bibenum numbers continuously throughout. To reset the
%   counter, use
%
%   \restartlist{bibenum}
%
%   at the place where you want the numbering to reset.

\makeatletter
\newlength{\bibhang}
\setlength{\bibhang}{1em}
\newlength{\bibsep}
 {\@listi \global\bibsep\itemsep \global\advance\bibsep by\parsep}
\newlist{bibsection}{itemize}{3}
\setlist[bibsection]{label=,leftmargin=\bibhang,%
        itemindent=-\bibhang,
        itemsep=\bibsep,parsep=\z@,partopsep=0pt,
        topsep=0pt}
\newlist{bibenum}{enumerate}{3}
\setlist[bibenum]{label=[\arabic*],resume,leftmargin={\bibhang+\widthof{[999]}},%
        itemindent=-\bibhang,
        itemsep=\bibsep,parsep=\z@,partopsep=0pt,
        topsep=0pt}
\let\oldendbibenum\endbibenum
\def\endbibenum{\oldendbibenum\vspace{-.6\baselineskip}}
\let\oldendbibsection\endbibsection
\def\endbibsection{\oldendbibsection\vspace{-.6\baselineskip}}
\makeatother

%% Reference the last page in the page number
%
% NOTE: comment the +LP line and uncomment the -LP line to have page
%       numbers without the ``of ##'' last page reference)
%
% NOTE: uncomment the \pagestyle{empty} line to get rid of all page
%       numbers (make sure includefoot is commented out above)
%
\usepackage{gitinfo} % Added by KY, for displaying git commit hash in
                     % CV document
\usepackage{fancyhdr,lastpage}
\pagestyle{fancy}
%\pagestyle{empty}      % Uncomment this to get rid of page numbers
\fancyhf{}\renewcommand{\headrulewidth}{0pt}
\fancyfootoffset{\marginparsep+\marginparwidth}
\newlength{\footpageshift}
\setlength{\footpageshift}
          {0.5\textwidth+0.5\marginparsep+0.5\marginparwidth-2in}
\lfoot{\hspace{\footpageshift}%
       \parbox{4in}{\, \hfill %
                    \arabic{page} of \protect\pageref*{LastPage} % +LP
%                    \arabic{page}                               % -LP
                    \hfill \,}} 

%KY added abbreviated git commit hash and ISO timestamp to footer
\rhead{Kristine M. Yu, CV, updated {\normalsize \gitAuthorDate}}


% Finally, give us PDF bookmarks
\usepackage{color,hyperref}
\definecolor{darkblue}{rgb}{0.145,0.204,0.58}
%\definecolor{darkred}{rgb}{0.843, 0.188, 0.122}
%\definecolor{darkblue}{rgb}{0.192,0.51,0.741}
\hypersetup{colorlinks,breaklinks,
            linkcolor=darkblue,urlcolor=darkblue,
            anchorcolor=darkblue,citecolor=darkblue}

% KY added macro for allowing footnotemarks to refer to same footnote
% https://tex.stackexchange.com/questions/10102/multiple-references-to-the-same-footnote-with-hyperref-support-is-there-a-bett

\newcommand{\footlabel}[2]{%
    \addtocounter{footnote}{1}%
    \footnotetext[\thefootnote]{%
        \addtocounter{footnote}{-1}%
        \refstepcounter{footnote}\label{#1}%
        #2%
    }%
    $^{\ref{#1}}$%
}

\newcommand{\footref}[1]{%
    $^{\ref{#1}}$%
}

%%%%%%%%%%%%%%%%%%%%%%%% End Document Setup %%%%%%%%%%%%%%%%%%%%%%%%%%%%


%%%%%%%%%%%%%%%%%%%%%%%%%%% Helper Commands %%%%%%%%%%%%%%%%%%%%%%%%%%%%

%%% HEADING AT TOP OF CURRICULUM VITAE

% The title (name) with a horizontal rule under it
% (optional argument typesets an object right-justified across from name
%  as well)
%
% Usage: \makeheading{name}
%        OR
%        \makeheading[right_object]{name}
%
% Place at top of document. It should be the first thing.
% If ``right_object'' is provided in the square-braced optional
% argument, it will be right justified on the same line as ``name'' at
% the top of the CV. For example:
%
%       \makeheading[\emph{Curriculum vitae}]{Your Name}
%
% will put an emphasized ``Curriculum vitae'' at the top of the document
% as a title. Likewise, a picture could be included:
%
%   \makeheading[{\includegraphics[height=1.5in]{my_picture}}]{Your Name}
%
% the picture will be flush right across from the name. For this example
% to work, make sure the extra set of curly braces is included. Also
% makes ure that \usepackage{graphicx} is somewhere in the preamble.
\newcommand{\makeheading}[2][]%
        {\hspace*{-\marginparsep minus \marginparwidth}%
         \begin{minipage}[t]{\textwidth+\marginparwidth+\marginparsep}%
             {\large \bfseries #2 \hfill #1}\\[-0.15\baselineskip]%
                 \rule{\columnwidth}{1pt}%
         \end{minipage}}

%%% SECTION HEADINGS

% The section headings. Flush left in small caps down pseudo-margin.
%
% Usage: \section{section name}
\renewcommand{\section}[1]{\pagebreak[3]%
    \vspace{1.3\baselineskip}%
    \phantomsection\addcontentsline{toc}{section}{#1}%
    \noindent\llap{\scshape\smash{\parbox[t]{\marginparwidth}{\hyphenpenalty=10000\raggedright #1}}}%
    \vspace{-\baselineskip}\par}

%%% LISTS

% This macro alters a list by removing some of the space that follows the list
% (is used by lists below)
\newcommand*\fixendlist[1]{%
    \expandafter\let\csname preFixEndListend#1\expandafter\endcsname\csname end#1\endcsname
    \expandafter\def\csname end#1\endcsname{\csname preFixEndListend#1\endcsname\vspace{-0.6\baselineskip}}}

% These macros help ensure that items in outer-type lists do not get
% separated from the next line by a page break
% (they are used by lists below)
\let\originalItem\item
\newcommand*\fixouterlist[1]{%
    \expandafter\let\csname preFixOuterList#1\expandafter\endcsname\csname #1\endcsname
    \expandafter\def\csname #1\endcsname{\let\oldItem\item\def\item{\pagebreak[2]\oldItem}\csname preFixOuterList#1\endcsname}
    \expandafter\let\csname preFixOuterListend#1\expandafter\endcsname\csname end#1\endcsname
    \expandafter\def\csname end#1\endcsname{\let\item\oldItem\csname preFixOuterListend#1\endcsname}}
\newcommand*\fixinnerlist[1]{%
    \expandafter\let\csname preFixInnerList#1\expandafter\endcsname\csname #1\endcsname
    \expandafter\def\csname #1\endcsname{\let\oldItem\item\let\item\originalItem\csname preFixInnerList#1\endcsname}
    \expandafter\let\csname preFixInnerListend#1\expandafter\endcsname\csname end#1\endcsname
    \expandafter\def\csname end#1\endcsname{\csname preFixInnerListend#1\endcsname\let\item\oldItem}}

% An itemize-style list with lots of space between items
%
% Usage:
%   \begin{outerlist}
%       \item ...    % (or \item[] for no bullet)
%   \end{outerlist}
\newlist{outerlist}{itemize}{3}
    \setlist[outerlist]{label=\enskip\textbullet,leftmargin=*}
    \fixendlist{outerlist}
    \fixouterlist{outerlist}

% An environment IDENTICAL to outerlist that has better pre-list spacing
% when used as the first thing in a \section
%
% Usage:
%   \begin{lonelist}
%       \item ...    % (or \item[] for no bullet)
%   \end{lonelist}
\newlist{lonelist}{itemize}{3}
    \setlist[lonelist]{label=\enskip\textbullet,leftmargin=*,partopsep=0pt,topsep=0pt}
    \fixendlist{lonelist}
    \fixouterlist{lonelist}

% An itemize-style list with little space between items
%
% Usage:
%   \begin{innerlist}
%       \item ...    % (or \item[] for no bullet)
%   \end{innerlist}
\newlist{innerlist}{itemize}{3}
    \setlist[innerlist]{label=\enskip\textbullet,leftmargin=*,parsep=0pt,itemsep=0pt,topsep=0pt,partopsep=0pt}
    \fixinnerlist{innerlist}

% An environment IDENTICAL to innerlist that has better pre-list spacing
% when used as the first thing in a \section
%
% Usage:
%   \begin{loneinnerlist}
%       \item ...    % (or \item[] for no bullet)
%   \end{loneinnerlist}
\newlist{loneinnerlist}{itemize}{3}
    \setlist[loneinnerlist]{label=\enskip\textbullet,leftmargin=*,parsep=0pt,itemsep=0pt,topsep=0pt,partopsep=0pt}
    \fixendlist{loneinnerlist}
    \fixinnerlist{loneinnerlist}

%%% EXTRA SPACE

% To add some paragraph space between lines.
% This also tells LaTeX to preferably break a page on one of these gaps
% if there is a needed pagebreak nearby.
\newcommand{\blankline}{\quad\pagebreak[3]}
\newcommand{\halfblankline}{\quad\vspace{-0.5\baselineskip}\pagebreak[3]}

%%% FORMATTING MACROS

% Provides a linked \doi{#1} that links doi:#1 to http://dx.doi.org/#1
\usepackage{doi}
% To change the text before the DOI, adjust this command
%\renewcommand\doitext{doi:}

% Provides a linked \url{#1} that doesn't require escape characters
\usepackage{url}

% You can adjust the style \url{} uses here:
% (options are: same, rm, sf, tt; defaults to tt)
\urlstyle{same}

% For \email{ADDRESS}, links ADDRESS to the url mailto:ADDRESS
% (uncomment to typeset the e\-/mail address in typewriter font;
%  otherwise, will be typeset in the \urlstyle above)
%\DeclareUrlCommand\emaillink{\urlstyle{tt}}
\providecommand*\emaillink[1]{\nolinkurl{#1}}
\providecommand*\email[1]{\href{mailto:#1}{\emaillink{#1}}}

\providecommand\BibTeX{{B\kern-.05em{\sc i\kern-.025em b}\kern-.08em \TeX}}
\providecommand\Matlab{\textsc{Matlab}}

% Custom hyphenation rules for words that LaTeX has trouble with
\hyphenation{bio-mim-ic-ry bio-in-spi-ra-tion re-us-a-ble pro-vid-er Media-Wiki}

%%%%%%%%%%%%%%%%%%%%%%%% End Helper Commands %%%%%%%%%%%%%%%%%%%%%%%%%%%

%%%%%%%%%%%%%%%%%%%%%%%%% Begin CV Document %%%%%%%%%%%%%%%%%%%%%%%%%%%%

\begin{document}
\makeheading{Kristine M.~Yu}

\section{Contact Information}

% NOTE: Mind where the & separators and \\ breaks are in the following
%       table. Table is one row made up of three parboxes. The left
%       parbox has address info, the middle parbox has a vertical bar,
%       and the right parbox has phone and electronic contact
%       information.
%
% MACROS: \rcollength is the width of the right column of the table
%             (adjust it to your liking; default is 1.85in).
%         \spacewidth is width of area between left and right boxes.
%
\newlength{\rcollength}\setlength{\rcollength}{2in}%
\newlength{\spacewidth}\setlength{\spacewidth}{20pt}
%
\begin{tabular}[t]{@{}p{\textwidth-\rcollength-\spacewidth}@{}p{\spacewidth}@{}p{\rcollength}}%

% Address box
\parbox{\textwidth-\rcollength-\spacewidth}{%
Associate Professor\\
\href{http://www.umass.edu/linguistics}{Department of Linguistics}\\
\href{http://www.umass.edu}{University of Massachusetts Amherst}\\
Integrative Learning Center\\
650 North Pleasant Street\\
Amherst, MA 01003-1100, USA}

&
% Uncomment to add a vertical bar in middle of contact information
%{\vrule width 0.5pt}
\parbox[m][5\baselineskip]{\spacewidth}{} &

% Non-snail-mail contact information
\parbox{\rcollength}{%
\textit{Phone:} +1-413-545-0885 \\
\textit{Fax:} +1-413-545-2792 \\
\textit{E-mail:} \email{krisyu@linguist.umass.edu}\\
\textit{WWW:} \href{http://www.krisyu.org/}{www.krisyu.org}}

\end{tabular}

%%
%% In modern CV's, it seems like ``Objective'' is frowned upon. Instead,
%% incorporate it into a well-constructed cover letter. The ``More
%% information'' can go at the end of the CV, but it should not distract
%% from the section giving references available to contact.
%%
%
% \section{Objective}
%
% Placement in an academic position (i.e., faculty, postdoctoral, or
% research scientist) that allows for advanced research in distributed
% complex adaptive systems (i.e., modeling, analysis, design, and
% verification) with a particular focus on the control of engineered
% agents (e.g., for communications, control, software, electronics, and
% sustainability) and the analysis of biological phenomena (e.g.,
% self-organization, ecological rationality)
% \begin{innerlist}
% \item More information and auxiliary documents can be found at\\\url{http://www.tedpavlic.com/facjobsearch/}
% \end{innerlist}

\section{Central Research Interests}

\textbf{Prosody, from the speech signal on up:}
tone, intonation, phonetics, phonology, mathematical modeling of
speech, language learning and variation, human language processing

\section{Academic Appointments}

\textbf{Assistant Professor} \hfill {January 2019 -- present}
\begin{innerlist}

    \item[] \href{http://www.umass.edu/linguist}{Department of Linguistics},
            \href{http://www.umass.edu/linguist}{University of Massachusetts}
    \begin{innerlist}
        \item Affiliations:
            \begin{innerlist}[\enskip$\circ$,leftmargin=*]
                \item {Five Colleges Prosody Group}, co-director
                \item {UMass Phonetics Lab}
                \item \href{https://blogs.umass.edu/comphon/}{UMass Computational Phonology Lab}
                \item \href{https://blogs.umass.edu/cogsci/}{UMass
                    Initiative for Cognitive Science}
                \item\href{http://www.cssi.umass.edu/}{UMass
                    Computational Social Science Initiative}, faculty associate
                \item\href{https://www.umass.edu/diversitysciences/}{UMass Institute of Diversity Sciences}, faculty affiliate
                \item\href{https://ds.cs.umass.edu/}{Center for Data
                  Science}, Five College faculty
            \end{innerlist}
    \end{innerlist}

\end{innerlist}

\halfblankline

\textbf{Assistant Professor} \hfill {September 2011 -- December 2018}
\begin{innerlist}

    \item[] \href{http://www.umass.edu/linguist}{Department of Linguistics},
            \href{http://www.umass.edu/linguist}{University of Massachusetts}
    \end{innerlist}

\halfblankline

\textbf{Postdoctoral Researcher} \hfill {September 2011 -- June 2012}
\begin{innerlist}

    \item[] \href{http://ling.umd.edu/}{Department of Linguistics},
            \href{http://www.umd.edu}{University of Maryland College
              Park}
            \begin{innerlist}
        \item Supervisor: Bill Idsardi
        \end{innerlist}

\end{innerlist}

\section{Education}

\href{http://www.ucla.edu}{\textbf{University of California Los Angeles}},
Los Angeles, CA
\begin{outerlist}

\item[] Ph.D.,
        \href{http://www.linguistics.ucla.edu}
             {Linguistics},
             September 2011
        \begin{innerlist}
        \item Thesis Topic: \href{http://www.krisyu.org/pages/pdfs/yu2011_diss.pdf}{The learnability of tones from the speech signal}
        \item Advisers:
              \href{http://www.linguistics.ucla.edu/people/stabler/}
                   {Professor Edward Stabler}, 
              \href{http://www.linguistics.ucla.edu/people/Sundara/}
                   {Professor Megha Sundara}
        \end{innerlist}

\item[] M.A.,
        \href{http://www.linguistics.ucla.edu}
             {Linguistics},
             December 2008
        \begin{innerlist}
        \item Thesis Topic: \href{http://www.linguistics.ucla.edu/general/matheses/Yu_UCLA_MA_2008.pdf}{The prosody of second position clitics and focus in Zagreb Croatian}
        \item Adviser:
              \href{http://www.linguistics.ucla.edu/people/jun/sun-ah.htm}
                   {Professor Sun-Ah Jun}
        \end{innerlist}

\end{outerlist}


\halfblankline


\href{http://www.uni-magdeburg.de/en/home-p-1.html}{\textbf{Otto-von-Guericke-Universit\"{a}t
  Magdeburg}},
Magdeburg, Germany
\begin{outerlist}

\item[] \href{http://www.fulbright.de/togermany/}{Fulbright scholar},
        \href{http://www.biophysik.ovgu.de/}
             {Abteilung Biophysik}, September 2004 -- August 2005
        \begin{innerlist}
        \item Synchronization of glycolytic oscillations in yeast cell populations
        \end{innerlist}

\end{outerlist}

\halfblankline

\href{http://www.stanford.edu}{\textbf{Stanford University}},
Stanford, CA
\begin{outerlist}

\item[] B.S.
        \href{http://chemistry.stanford.edu/}
             {Chemistry}, 
             June 2004
        \begin{innerlist}
        \item With distinction
        \item Minor in physics
        \end{innerlist}

\end{outerlist}

\halfblankline

\href{http://www.usc.edu}{\textbf{University of Southern California}},
Los Angeles, CA
\begin{outerlist}

\item[] 
        August 2000 -- May 2001
        \begin{innerlist}
        \item Trustee Scholar
        \item Resident Honors Program
        \end{innerlist}

\end{outerlist}


% \section{Submitted Journal Publications}
%
% % Add a little space to nudge next ``Ref'd Journal Publications'' marginpar
% % down to make room for tall ``Submitted Journal Publications''
% % marginpar. If there are enough submitted journal publications, this
% % space will not be needed (and should be removed).
% \vspace{0.1in}


\section{Honors}

\href{http://www.umass.edu/issr/index.php}{\textbf{University of Massachusetts
    Institute for Social Science Research}}
\begin{innerlist}
\item \href{http://www.umass.edu/issr/scholars/}
           {Social Science Research Scholar}, 2014 -- 2015
\end{innerlist}

\halfblankline

\href{http://www.nsf.gov/}{\textbf{National Science Foundation}}
\begin{innerlist}
\item \href{http://www.nsf.gov/grfp}
           {Graduate Research Fellowship}, 2008 -- 2011
\end{innerlist}

\halfblankline

\textbf{Ohio State University Research Foundation}
\begin{innerlist}
\item \href{http://www.cse.ohio-state.edu/mlss09/}{Machine Learning Summer School/Workshop 2009} Travel Grant
\end{innerlist}

\halfblankline

\href{http://www.ucla.edu}{\textbf{University of California Los Angeles}}
\begin{innerlist}
\item Department of Linguistics \href{http://www.linguistics.ucla.edu/general/LadefogedScholarship/}{Ladefoged Scholarship Award}, 2010
\item Graduate Research Mentorship Program Fellowship, 2007 -- 2008 
\item Summer Graduate Research Mentorship Program Fellowhip, 2007, 2008
\item Linguistic Society of America Linguistic Institute Fellowship, 2007
\item Humanities Dean's Fellowship, 2006 -- 2007
\end{innerlist}

\halfblankline

\href{http://www.iie.org/fulbright}{\textbf{Institute of International
    Education Fulbright Program}}
\begin{innerlist}
\item \href{http://www.fulbright.de/togermany/}
           {Fulbright Scholarship, Germany}, 2004 -- 2005
\item \href{http://www.fulbright.de/togermany/}
           {Fulbright Enterprise Scholarship, Germany} (4/107 grantees
           selected), 2004 -- 2005
\end{innerlist}

\halfblankline

\href{https://www.daad.org/}{\textbf{Deutscher Akademischer Austausch Dienst}}
\begin{innerlist}
\item \href{https://www.daad.org/?p=73215}
           {DAAD study scholarship}, 2004 (Declined)
\end{innerlist}

\halfblankline

\href{http://www.act.org/goldwater/}{\textbf{Barry Goldwater Scholarship and Excellence in Education Program}}
\begin{innerlist}
\item {Barry Goldwater scholarship}, 2003
\end{innerlist}

\section{Grants}


\textbf{External} \hfill 

\halfblankline


\begin{bibsection}

    \item Principle Investigator, \href{https://www.nsf.gov/awardsearch/showAward?AWD_ID=1749321&HistoricalAwards=false}{``Workshop on sociolectal and dialectal prosodic variation: Amherst, MA - October 11-13, 2018''}, National Science Foundation, Linguistics Program and Documenting Endangered Languages program, BCS Award 1749321. \$35,873.00, May 1,~2018 -- October 31,~2019. For \href{https://etap4.krisyu.org}{Experimental and Theoretical Advances in Prosody 4 (ETAP4)}.
  
  
    \item Consultant,
        ``Effects of Syntactic Constituency on Phonology and Phonetics
        of Tone'' \linebreak[4] (PI: Lisa Selkirk), National Science Foundation, Linguistics Program, BCS Award
        \href{http://www.nsf.gov/awardsearch/showAward?AWD_ID=1147083}{1147083}.
        \$361,283.00, July 1,~2012 -- December 31,~2015.

\end{bibsection}

\blankline

\blankline

\textbf{Internal} \hfill 

\halfblankline

\begin{bibsection}

     \item Principle Investigator, UMass \href{https://www.umass.edu/research/faculty-research-granthealey-endowment-grant-frgheg}{Faculty Research Grant/Healey Endowment Grant}. ``The sound of \textit{been} in real-time comprehension of African American English speech''.  \$14,965, 2018 -- 2019.

    \item Co-PI, UMass Institute of Diversity Sciences \href{https://www.umass.edu/diversitysciences/seed-grants}{Seed Grant}. ``Processing Non-native Speech in Noisy Classrooms''. PI: Lisa Sanders, Co-PIs: Meghan Armstrong-Abrami, Anne Gilman. \$12,000, 2018 -- 2019.

       
    \item Principle Investigator, UMass Institute of Diversity Sciences \href{https://www.umass.edu/diversitysciences/seed-grants}{Seed Grant}. ``Discovering African American English Speech Melodies''. Co-PIs: Meghan Armstrong-Abrami, Lisa Green, Brendan O'Connor. \$12,000, 2018 -- 2019.
  
    \item Principle Investigator, Andrew Mellon Foundation / Center for Teaching \& Faculty
      Development at the University of Massachusetts Amherst, ``Five
      Colleges Prosody Community'', \$10,000, 2013 -- 2014.

        \begin{innerlist}
        \item Mentoring network small group lunches
        \item Winter prosody bootcamp (regional): January 17-18, 2014
        \item UMass Tone Workshop (international): June 2-3, 2014
        \end{innerlist}
\end{bibsection}
\halfblankline

\section{Refereed Journal Publications}
\restartlist{bibenum}

\textbf{Linguistics} \hfill 

\halfblankline

\begin{bibenum}
      \item Yu, Kristine M. and Edward Stabler. (2017).
        (In)variability in the Samoan syntax/prosody interface and consequences for
        syntactic parsing. \emph{Laboratory Phonology}. [\href{https://www.journal-labphon.org/articles/10.5334/labphon.113/}{open access pub}, \href{https://doi.org/10.5334/labphon.113.s1}{supplementary materials}]
    \item Yu, Kristine M. (2017) The role of time in phonetic
      spaces: temporal resolution in Cantonese tone
      perception. \emph{Journal of Phonetics}. [\href{https://doi.org/10.1016/j.wocn.2017.06.004}{open access pub}, \href{https://github.com/krismyu/resolution}{supplementary material}]
    \item \textbf{Yu, Kristine M.} and Hiu Wai Lam. (2014). The role of
      creaky voice in Cantonese tone perception. \emph{The Journal of
        the Acoustical Society of America}, 136(3), 1320-1333.
      [\href{http://scitation.aip.org/content/asa/journal/jasa/136/3/10.1121/1.4887462}{web},
      \href{http://www.krisyu.org/pages/pdfs/yulam2014-jasa-cantcr.pdf}{pdf},
      \href{http://www.krisyu.org/blog/supp-material-cantonese-creak-perception.html}{supplementary materials}]

    \item Yu, Kristine M. (2014). The experimental state of mind in
      elicitation: illustrations from tonal fieldwork. \emph{Language Documentation \&
        Conservation}, 8, 738--777. 
      [\href{http://scholarspace.manoa.hawaii.edu/bitstream/handle/10125/24623/Yu.pdf?sequence=1}{web},
      \href{http://www.krisyu.org/pages/pdfs/yu2014-kiy.pdf}{pdf}, 
      \href{http://www.krisyu.org/blog/ldc-kiy-overview.html}{supplementary materials}]

    \item Zuraw, Kie, \textbf{Yu, Kristine M.}, and Robyn Orfitelli. (2014). The word-level prosody of
      {S}amoan. \emph{Phonology}, 31(2), 271--327.
      [\href{http://dx.doi.org/10.1017/S095267571400013X}{web},
      \href{http://www.krisyu.org/pages/pdfs/zurawyuorfitelli2014.pdf}{pdf}]

\end{bibenum}

\blankline

\blankline

\textbf{Biochemistry and genetics, prior to change of field to linguistics} \hfill 

\halfblankline

\begin{bibenum}
    \item  J. Liu, S. Louie, W. Hsu, \textbf{K. M.~Yu},
      H. B. Nicholas, and G. L. Rosenquist. (2008). Tyrosine
        sulfation is prevalent in human chemokine receptors important
        in lung disease. \emph{American Journal of Respiratory Cell and Molecular Biology}, 38:
      738--743. \doi{10.1165/rcmb.2007-0118OC} 

      \item Y. R. Thorstenson, A. Roxas, R. Kroiss, M. A. Jenkins,
        \textbf{K. M.~Yu}, T. Bachrich, D.  Muhr, T. L. Wayne, G. Chu,
        R. W. Davis, T. M. U. Wagner, and P. J. Oefner. (2003). 
        Contributions of \textit{ATM} mutations to familial breast and
        ovarian cancer. \emph{Cancer Research}, 63:
        3325--3333. \url{http://cancerres.aacrjournals.org/cgi/content/abstract/63/12/3325}  

    \item A. S. Torralba, \textbf{K. Yu}, P. Shen, P. J. Oefner, and
      J. Ross. (2003). Experimental
        test of a method for determining causal connectivities of
        species in reactions. \textit{Proc. Natl. Acad. Sci. USA},
      100: 1494--1498. \doi{10.1073/pnas.262790699}

    \item \textbf{K. M.~Yu}, J. Liu, R. Moy, H. C. Lin,
      H. B. Nicholas, and G. L Rosenquist. (2002). Prediction of
      tyrosine sulfation in seven-transmembrane peptide receptors.
    \emph{Endocrine}, 19: 333--337. \doi{10.1385/ENDO:19:3:333} 

    \item S. \v{S}ale, R. Sung, P. Shen, \textbf{K. Yu}, Y. Wang,
      G. Duran, J. Kim, T. Fojo, P. J. Oefner, and
      B. I. Sikic. (2002). Conservation
        of the class I $\beta$-tubulin gene in human populations and
        lack of mutations in lung cancers and paclitaxel-resistant
        ovarian cancers. \emph{Molecular Cancer Therapeutics}, 1:
        215--225. \url{http://mct.aacrjournals.org/content/1/3/215.short} 

\end{bibenum}

% Add a little space to nudge next ``Conference Publications'' marginpar
% down to make room for tall ``Submitted Conference Publications''
% marginpar. If there are enough submitted journal publications, this
% space will not be needed (and should be removed).
\vspace{0.1in}

\section{Refereed Book Chapters}

\restartlist{bibenum}
\begin{bibenum}
    \item Yu, Kristine M. (2019). Parsing with Minimalist Grammars and prosodic trees. In \emph{Minimalist parsing}, ed.\ Robert C. Berwick and Edward Stabler.  
    \item Kuo, Grace C.-H. and \textbf{Kristine M.~Yu}. (2012). Taiwan Mandarin
 quantifiers. In \emph{Handbook of Quantifiers in Natural  Language},
 ed.\ Edward Keenan and Denis Paperno, pp.\ 647 --
 698. \doi{10.1007/978-94-007-2681-9_12} [\href{http://www.krisyu.org/pages/pdfs/kuo-yu-taiwan-mandarin-quantifiers.pdf}{pdf}] 

\end{bibenum}

\section{Paper-reviewed Conference Proceedings Papers}

\restartlist{bibenum}
\begin{bibenum}
  
\item Yu, Kristine M. (2018). Advantages of constituency: computational perspectives on word prosody in Samoan. In: \emph{Formal grammar: 22nd and and 23rd international conferences}, ed.\ Annie Foret, Reinhard Muskens, and Sylvain Pogodalla, Springer-Verlag: Berlin/Heidelberg. [\href{https://github.com/krismyu/smo-constituency-feet/blob/master/fg-kmyu.pdf}{pdf}]

  \item Kimper, Wendell, \textbf{Yu, Kristine}, Bennett, William, and Christopher Green. (To appear). Acoustic correlates of harmonic classes in
    Somali. \emph{Proceedings of the 47th Annual Conference on African Linguistics, University of California Berkeley}.
  
    \item Yu, Kristine M. (2011). Temporal resolution in {C}antonese tone
      perception. In: \emph{Proceedings of The Psycholinguistic Representation of Tone}.

    \item \textbf{Yu, Kristine M.} and Hiu Wai Lam. (2011). The role of creaky
      voice in {C}antonese tonal perception. In: \emph{Proceedings of the
        17th International Congress of Phonetic Sciences},
      2240 -- 2243. [\href{http://www.krisyu.org/pages/pdfs/yu-lam2011-icphs-cantonese-creak.pdf}{pdf}] 

    \item Shue, Yen-Liang, Keating, Patricia, Vicenik, Chad, and \textbf{Kristine
      Yu}. (2011). VoiceSauce: a program for voice analysis. In:
      \emph{Proceedings of the 17th International Congress of Phonetic
        Sciences}
      1846 -- 1849. [\href{http://www.krisyu.org/pages/pdfs/shueETAL2011-icphsxvii-voicesauce.pdf}{pdf}]

    \item Yu, Kristine M. Laryngealization and features for Chinese
      tonal recognition. (2010). In: \emph{Proceedings of Interspeech 2010},
      1529 --
      1532. [\href{http://www.isca-speech.org/archive/interspeech_2010/i10_1529.html}{web}, \href{http://www.krisyu.org/pages/pdfs/yu2010-interspeech-laryngealization.pdf}{pdf}]

    \item Yu, Kristine M. (2009). The prosody of second position
      clitics and focus in Zagreb Croatian. In: Maria Babyonyshev,
      Daria Kavitskaya, Jodi Reich, eds., \emph{Formal Approaches to Slavic Linguistics \#17: 
The Yale Meeting 2008.}, Michigan Slavic Publications, Ann Arbor, MI,
pps.\ 253 -- 270. [\href{http://www.krisyu.org/pages/pdfs/yu2009-fasl17-croatian-2pc.pdf}{pdf}]  

\end{bibenum}


\section{Abstract-reviewed Conference Proceedings Papers}

\restartlist{bibenum}
\begin{bibenum}

    \item \textbf{Yu, Kristine M.} and Deniz
      \"{O}zy\i{}ld\i{}z. (2016). The absolutive \textit{ia} particle in Samoan. In: \emph{Proceedings of the 42nd Annual
      Meeting of the Berkeley Linguistic Society}. [\href{http://linguistics.berkeley.edu/bls/previous_proceedings/bls42.pdf}{web}, \href{http://www.krisyu.org/pages/pdfs/yuozyildiz2016.pdf}{pdf}]

    \item Yu, Kristine M. (2011). The sound of ergativity:
      morphosyntax-prosody mapping in Samoan. In: Suzi Lima, Kevin
      Mullin, Brian Smith, eds., \emph{Proceedings of the 39th Annual
      Meeting of the North East Linguistic Society}, Vol.\ 2, Graduate
    Linguistic Student Association, Amherst,
    MA, pps.\ 825 -- 838. [\href{http://www.krisyu.org/pages/pdfs/yu2011-nels39-samoan.pdf}{pdf}]

\end{bibenum}


\section{Manuscripts under Review}
\restartlist{bibenum}

\begin{bibenum}
    \item Yu, Kristine M., ud Dowla Khan, Sameer, and Megha
      Sundara. (under revision). Intonational phonology in Bengali and English
      infant-directed speech. \emph{Laboratory Phonology}.
    \item Yu, Kristine M. (under review). Tonal marking of absolutive case
      in Samoan. \emph{Natural Language \& Linguistic Theory}.
\end{bibenum}


\blankline

\section{Technical Reports}

\restartlist{bibenum}
\begin{bibenum}

  \item Yu, Kristine M. (2010). Representational maps from the speech signal to
phonological categories: a case study with lexical tones. In: Thomas
Graf, ed., {UCLA} Working Papers in Linguistics, Papers in Mathematical Linguistics 1,
Volume 15, Article 5: 1 -- 30. [\href{http://www.linguistics.ucla.edu/faciliti/wpl/issues/wpl15/papers/15-5.pdf}{pdf}]

\end{bibenum}

\section{Popular\linebreak[4]Articles}

\restartlist{bibenum}
\begin{bibenum}

\item \textbf{Kristine M.~Yu}, Sameer ud Dowla Khan, Alejandrina Cristia, Huei-Mei
Liu, Megha Sundara (2011). \href{http://acoustics.org/pressroom/httpdocs/162nd/Yu_5aSCa3.html}{The
  joint purpose of pitch patterns in "motherese": drawing attention
  and conveying language structure.} ASA Lay Language Papers 162nd
Acoustical Society of America Meeting. On-line at: \url{http://acoustics.org/pressroom/httpdocs/162nd/Yu_5aSCa3.html} 

\end{bibenum}

\section{Invited Talks}

\restartlist{bibenum}
\begin{bibenum}

\item Yu, Kristine M. (2018). Linguistic structure in the speech melodies of speech to infants. National Taiwan University, Department of Foreign Languages \& Literatures. Taipei, Taiwan. May 16, 2018. 
  
\item Yu, Kristine M. (2018). Recovering sentence structure from speech melodies. National Chiao Tung University, Department of Foreign Languages and Literatures. Hsinchu, Taiwan. May 14, 2018.
  
\item Yu, Kristine M. (2017). Parsing with prosody: towards a computational model of prosodically-informed syntactic parsing in Samoan
. Brown University, Department of
  Cognitive, Linguistic, and Psychological Sciences. Michael S. Goodman '74 Memorial Seminar Series. April 19, 2017.  
\item Yu, Kristine M. (2017). Towards a model of the syntax-prosody interface in Samoan. Yale University, Department of
  Linguistics Colloquium Series. January 23, 2017. [\href{https://speakerdeck.com/krisyu/towards-a-model-of-the-samoan-syntax-prosody-interface}{slides}]
  \item Yu, Kristine. (2016). Commentary on Elfner (2016), Strong
    Start and phonological phrasing in Irish. International workshop on the effects of
    constituency on sentence phonology. University of Massachusetts
    Amherst, July 29, 2016. [\href{https://speakerdeck.com/krisyu/commentary-on-elfners-strong-start-and-phonological-phrasing-in-irish}{slides}]

  \item Yu, Kristine. (2016). Prosodic constituency in Samoan. International workshop on the effects of
    constituency on sentence phonology. University of Massachusetts
    Amherst, July 30, 2016. [\href{https://speakerdeck.com/krisyu/prosodic-constituency-in-samoan}{slides}]

  \item Yu, Kristine. (2016). The learnability of tones from the
    speech signal. Data Science Tea, Center for Data
    Science, University of Massachusetts Amherst, April 26, 2016.

  \item Yu, Kristine. (2015). Linguistic tone and the input to
    computational models of sentence comprehension. Computational
    Social Science Initiative Lunch Seminar, University of
    Massachusetts Amherst, November 13, 2015.

  \item Yu, Kristine. (2015). Tone in parsing. Workshop on Minimalist
    Parsing, Massachusetts Institute of Technology. October 9--11, 2015.

    \item Yu, Kristine M. (2015). Grammatical tones in
      Samoan. University of Auckland, Applied Language Studies and Linguistics, School of Cultures, Languages and Linguistics, Pacific Studies. July 2015.

    \item Yu, Kristine M. (2015). Tonal marking of absolutive case in Samoan. Massachusetts Institute of
      Technology Linguistics Department Colloquium. February 2015.

    \item Yu, Kristine M. (2014). Tonal marking of absolutive case in
      Samoan. University of California Santa Cruz Phlunch. October 2014.

    \item Yu, Kristine M. (2014). The sound of ergativity: tonal
      marking of absolutive case in Samoan. McGill University Workshop
      on Syntax-Prosody Mapping in Verb-Initial Languages. May 2014. [\href{https://speakerdeck.com/krisyu/tonal-marking-of-absolutive-case-in-samoan}{slides}]

    \item Yu, Kristine M. (2013). Phonological structure in infant
      directed speech. Stony Brook University Linguistics Colloquium. September 2013.

    \item Yu, Kristine M. (2012). The learnability of tones from the
      speech signal. New York University Linguistics Colloquium. November 30, 2012. 

    \item Yu, Kristine M. (2012). Samoan morphosyntax-prosody
      mapping. University of Maryland Linguistics Colloquium. February
      3, 2012.

    \item Yu, Kristine M. (2011). The learnability of tones from the
      speech signal. Macquarie University Linguistics
      Colloquium. November 30, 2011. [\href{https://speakerdeck.com/krisyu/the-learnability-of-the-tones-from-the-speech-signal}{slides}]

    \item Yu, Kristine M. (2011). Speech rhythm and
      acquisition. Macquarie University Child Language Lab. November 29, 2011. 

    \item Yu, Kristine M. (2011). The learnability of tones from the
      speech signal. Harvard University Linguistics
      Colloquium. November 15, 2011.

    \item Yu, Kristine M. (2011). The learnability of tones from the
      speech signal. University of California San Diego Linguistics
      Colloquium. November 7, 2011. 

    \item Yu, Kristine M. (2011) The learnability of language-specific
      f0-based rhythmic patterns. 162nd meeting of the Acoustical
      Society of America. November 4, 2011. 

    \item Yu, Kristine M. (2011). The learnability of tones from the
      speech signal. University of Delaware Linguistics and Cognitive
      Science Colloquium. October 7, 2011. 

    \item Yu, Kristine M. (2011). Studying language learning from the
      speech signal. Institute for Computational and Experimental
      Study of Language Launch, University of Massachusetts, Amherst, April 1, 2011.

    \item Yu, Kristine M. (2011). How to study how tones are learned. Workshop on how to
      study a tone language: from the first elicitation to the latest
      software. University of California, Berkeley, February 21, 2011.

    \item Yu, Kristine M. (2011). Inductive biases for learning
      phonological categories. University of Massachusetts, Amherst, February 15, 2011. 

    \item Yu, Kristine M. (2011). Discovering the intonational
      phonology of Samoan: a fieldwork tutorial. University of
      Massachusetts, Amherst, February 14, 2011.  

    \item Yu, Kristine M. (2009). The sound of ergativity: morphosyntax-prosody mapping in
      Samoan. University of Chicago Workshop on Language, Cognition,
      and Computation, June 5, 2009.

\end{bibenum}

\section{Refereed Conference and Workshop Presentations}

\restartlist{bibenum}
\begin{bibenum}

\item Yu, Kristine. M, ud Dowla Khan, Sameer, and Megha Sundara. (2018) Implementing Finite State Intonational Grammars to Understand Gradient Prosodic Manipulations. Hanyang International Symposium on Phonetics \& Cognitive Sciences of Language. May 18, 2018. Hanyang University. Seoul, Korea. [\href{https://speakerdeck.com/krisyu/implementing-finite-state-intonational-grammars-to-understand-gradient-prosodic-manipulations-in-infant-directed-speech}{slides}].
  
 \item Yu, Kristine M. (2018). Distinct kinds of tones in Samoan from spell-out and prosodic phrasing. The 25th Meeting of the Austronesian Formal Linguistics Association (AFLA 25). May 11, 2018. [\href{https://speakerdeck.com/krisyu/distinct-kinds-of- tones-in-samoan-from-spell-out-and-prosodic- phrasing}{slides}] 
  
  \item Kimper, Wendell, \textbf{Yu, Kristine}, Bennett, William, and Christopher Green. (2016). Acoustic correlates of harmonic classes in
    Somali. 47th Annual Conference on African Linguistics, University
    of California Berkeley, March 24, 2016

  \item \textbf{Yu, Kristine} and Deniz \"{O}zy\i{}ld\i{}z. (2016). Emergence of tonal absolutive case
    marking in Samoan. The 42nd Annual Meeting of the Berkeley
    Linguistics Society, February 5, 2016.

  \item Yu, Kristine. (2015). Tonal marking of absolutive case in
    Samoan. American International Morphology Meeting 3, University of
    Massachusetts Amherst, October 3, 2015.

  \item Yu, Kristine M. (2015). Tonal marking of absolutive case in
    Samoan. Morphosyntactic Triggers of Tone:
New Data and Theories, University of Leipzig, Germany, June 12, 2015.  

  \item Yu, Kristine M. and Edward Stabler. (2015). A parsing model for crowding, speech rate and syntax of tone. Experimental and Theoretical Approaches to Prosody 3,
    University of Illinois Urbana-Champaign, May 29, 2015.

  \item Yu, Kristine M. (2015). Tonal marking of absolutive case in
    Samoan. Austronesian Formal Linguistics Association 22, McGill
    University, Montreal, Canada, May 23, 2015.  

\item \textbf{Yu, Kristine M.}, Sameer ud Dowla Khan and Megha
  Sundara. (2014). Intonational phonology in Bengali and English
  infant-directed speech. Speech Prosody 7, Dublin, 2014. [\href{https://www.superlectures.com/speechprosody2014/intonational-phonology-in-bengali-and-english-infant-directed-speech}{video/slides}]
  \item Yu, Kristine M., and Hiu Wai Lam. (2011). The role of creaky voice in Cantonese
    tonal perception. International Congress of Phonetic Sciences XVII, Hong Kong, 2011.

  \item Yu, Kristine M. (2011). Representations for learning
    phonological categories. Linguistics Society of America,
    Pittsburgh, PA, January 7, 2011.  

  \item Yu, Kristine M. (2011). Coarse representations for phonemic
    categories. North Eastern Linguistic Society 41, Philadelphia, PA,
    October 23, 2010. 

  \item Yu, Kristine M. (2009). The sound of ergativity: morphosyntax-prosody mapping in
    Samoan. Austronesian Formal Linguistics Association XVI, UC Santa Cruz, CA.

  \item Yu, Kristine M. (2009). The sound of ergativity: morphosyntax-prosody mapping in
    Samoan. Linguistics Society of America, San Francisco, CA.

  \item Yu, Kristine M. (2008). The sound of ergativity: morphosyntax-prosody mapping in
    Samoan. North East Linguistic Society 39, Ithaca, NY. 

  \item Zuraw, Kie, Orfitelli, Robyn, and \textbf{Kristine
    Yu}. (2008). Word-level prosody of Samoan. Austronesian Formal
    Linguistics Association XV, Sydney, Australia.  

  \item Yu, Kristine. (2008). The prosody of second position clitics
    and focus in Zagreb Croatian. Formal Approaches to Slavic
    Linguistics 17, New Haven, CT, May 11, 2011. 

\end{bibenum}

\section{Refereed Conference Posters}

\restartlist{bibenum}
\begin{bibenum}

\item Yu, Kristine M. (2016). Temporal aspects of talker variability in lexical tones. Higher-order structure in speech variability: phonetic/phonological covariation and talker adaptation
, Satellite Workshop of the 15th Conference on Laboratory
  Phonology. Cornell University, Ithaca, NY.

\item Yu, Kristine. (2016). The emergence of an inflectional edge tone
  morpheme in Samoan. The 15th Conference on Laboratory
  Phonology. Cornell University, Ithaca, NY.

\item Khan, Sameer ud Dowla and \textbf{Kristine M. Yu}. (2014). Intonational phonology
  in infant-directed speech. Linguistics Society of America, Minneapolis, MN.

\item Silverstein, Kate and \textbf{Kristine M. Yu}. (2013). Investigating
  tonal spaces using an extension of VoiceSauce voice analysis
  software. Acoustical Society of America, San Francisco, CA, December 2013.

\item \textbf{Yu, Kristine M.}, Khan, Sameer ud Dowla and Megha Sundara. (2011). A cross-linguistic
investigation of information structure in infant-directed
speech. Experimental and Theoretical Approaches to Prosody, September 23, 2011.

\item Khan, Sameer, \textbf{Yu, Kristine~M.}, and Jaime Roemer. (2011). Speech rhythm and f0 patterns
in Bengali: implications for prosodic acquisition. Acoustical
Society of America, Seattle, WA, May 27, 2011.

\item Yu, Kristine~M. (2011). Acoustic representations for tonal phonological
categories. Acoustical Society of America, Seattle, WA, May
23, 2011.

\item Yu, Kristine~M. (2010). Laryngealization and features for Chinese tonal
recognition. Interspeech 2010, Makuhari, Japan, September 14, 2010.

\item Yu, Kristine. (2010). Linear separability and feature selection in the
acquisition of tone. LabPhon 12, Albuquerque, NM, July 8, 2010.

\item Lam, Hiu Wai and \textbf{Kristine M.~Yu}. (2010). The role of creaky voice quality in {C}antonese
tonal perception. Acoustical Society of America, Baltimore,
MD, April 23, 2010.

\item \textbf{Yu, Kristine~M.}, Lam, Hiu Wai, and Shing-Yin Li. (2010). An acoustic and
electroglottographic study of {C}antonese tone. Acoustical
Society of America, Baltimore, MD, April 23, 2010.

\item Yu, Kristine~M. (2010). Linear separability and feature selection in the
acquisition of tone. Computational Modelling of Sound Pattern
Acquisition, University of Alberta, Feb. 14, 2010.

\item Yu, Kristine~M. (2009). Contextual tonal variation in level tone
languages. Acoustical Society of America, Portland, OR.

\item Orfitelli, Robyn and \textbf{Kristine Yu}. (2009). The intonational phonology of Samoan. Austronesian Formal Linguistics Association XVI, UC Santa Cruz, CA.

\item Yu, Kristine. (2008). The prosody of second position clitic placement and focus in Zagreb Croatian. Acoustics'08, Acoustical Society of America, Paris, France. 

\end{bibenum}

\section{Non-refereed Conference and Workshop Presentations}

\restartlist{bibenum}
\begin{bibenum}
    \item Yu, Kristine M. (2016). Advantages of constituency:
      computational perspectives on Samoan word prosody. Northeast
      Computational Phonology Circle. University of Massachusetts
      Amherst. September 24, 2016. [\href{https://speakerdeck.com/krisyu/advantages-of-constituency-computational-perspectives-on-samoan-word-prosody}{slides}]   

    \item Yu, Kristine M. (2011). The learnability of tones from the
      speech signal. Northeast Computational Phonology Circle. Yale
      University. October 15, 2011. [\href{https://speakerdeck.com/krisyu/the-learnability-of-tones-from-the-speech-signal}{slides}]  

\end{bibenum}

\section{Student Advising}

\textbf{Doctoral dissertations chair/co-chair} \hfill 

\begin{innerlist}
    \item Kusmer, Leland (in progress). Co-chair (with Kyle Johnson).
    \item Hauser, Ivy (in progress). Co-chair (with John Kingston).
\end{innerlist}

\halfblankline

\textbf{Doctoral dissertation committees, committee member} \hfill 

\begin{innerlist}
    \item \"{O}zy\i{}ld\i{}z, Deniz, in progress.
    \item Mullin, Kevin (2012-2017).
    \item S\'anchez Alvarado, Covadonga (external member, in progress).
    \item Nazarov, Aleksei (2016). \emph{Extending Hidden Structure Learning: Features, Opacity, and Exceptions}.
    \item Pizzo, Presley (2015). \emph{Investigating properties of
        phonotactic knowledge through web-based experimentation}.
\end{innerlist}

\halfblankline

\textbf{Comprehensive paper committees} \hfill 

\begin{innerlist}
  \item \"{O}zy\i{}ld\i{}z, Deniz (2017). \emph{Focus, and the factive inference with unembedded triggers}.
    \item Kusmer, Leland (2016). \emph{Prosody and the disjoint alternation in Tshivenda}. 
    \item Vostrikova, Ekaterina (2015, 2016). 
    \item Hauser, Ivy (2015). \emph{Dispersion (and lack thereof) in stop inventories}
    \item Conner, Tracy (2014). \emph{Questioning Inversion and
        Truncation as Constraints on Variation in African American English Polar Question Intonation}
    \item Fainleib, Yelena (2014). \emph{Evidence for Syllable Weight Sensitivity and Stratum Conditioning in Modern Hebrew Nouns Stress System.}
    \item Nazarov, Aleksei (2013). \emph{Generalization in phonological representations: a learning-based argument for grammars with multiple levels of representation}.
    \item Fainleib, Yelena (2013). \emph{Relative scaling of prosodic boundaries size in sentence production}. 
    \item Elsman, Minta (2013). \emph{Vowel devoicing and syllable position in Uzbek}.
    \item Moore-Cantwell, Claire (2012). \emph{Syntactic probability
        influences duration}.
    \item Also: Pizzo, Presley; Mullin, Kevin (2012).
\end{innerlist}

\halfblankline

\textbf{Undergraduate honors theses, chair/co-chair} \hfill 
\begin{innerlist}
    \item Higgins, Valerie (2017-2018). Linguistics/Spanish, Linguistics/Anthrophology. 
\end{innerlist}


\section{Teaching Experience}

\href{http://www.umass.edu}{\textbf{University of Massachusetts Amherst}},
Amherst, MA
\begin{outerlist}

\item[] \textit{Instructor}%
    \hfill \textbf{September~2012 -- present}
    \begin{innerlist}

        \item LINGUIST~197LM:
          Language and Music (Undergraduate)
        \begin{innerlist}[\enskip$\circ$,leftmargin=*]
            \item Spring~2017, Spring~2018
        \end{innerlist}

        \item LINGUIST~751:
          Topics in Phonology (Graduate)
        \begin{innerlist}[\enskip$\circ$,leftmargin=*]
            \item Fall~2015 (\emph{Somali field methods.})
        \end{innerlist}

        \halfblankline
        \item LINGUIST~414:
          Introduction to Phonetics for Linguistics (Undergraduate)
        \begin{innerlist}[\enskip$\circ$,leftmargin=*]
            \item Fall~2015, Fall~2016, Fall~2017, Spring~2019
        \end{innerlist}


        \halfblankline
        \item LINGUIST~592B:
          Speech processing (Undergraduate/Graduate)
        \begin{innerlist}[\enskip$\circ$,leftmargin=*]
            \item Spring~2014, Spring~2018
        \end{innerlist}

        \halfblankline

        \item LINGUIST~751:
          Topics in Phonology (Graduate)
        \begin{innerlist}[\enskip$\circ$,leftmargin=*]
            \item Spring~2014 (co-taught with Brian Dillon;
              \emph{Prosodic parsing})
        \end{innerlist}

        \halfblankline

        \item LINGUIST~716:
          Topics in Phonetics (Graduate)
        \begin{innerlist}[\enskip$\circ$,leftmargin=*]
            \item Fall~2013 (co-taught with John Kingston; \emph{Categories}), Fall~2019
        \end{innerlist}

        \halfblankline

        \item LINGUIST~606:
          Phonological Theory (Graduate)
        \begin{innerlist}[\enskip$\circ$,leftmargin=*]
            \item Spring~2013, Spring~2015, Spring~2017, Spring~2019
        \end{innerlist}

        \halfblankline

        \item LINGUIST~748:
          Structure of a non Indo-European Language (Graduate)
        \begin{innerlist}[\enskip$\circ$,leftmargin=*]
            \item Spring~2013 (co-taught with Alice Harris; Language: \href{http://www.ethnologue.com/language/dzo}{Dzongkha})
        \end{innerlist}

        \halfblankline

        \item LINGUIST~730:
          Proseminar Phonological Theory (Graduate)
        \begin{innerlist}[\enskip$\circ$,leftmargin=*]
            \item Fall~2012 (\emph{Prosodic theory})
            \item Fall~2017 (\emph{Foundations of prosody: back to the future})
        \end{innerlist}

    \end{innerlist}

\end{outerlist}

\section{Professional Service}

\textbf{Conference organization}

    \begin{innerlist}

    \item Co-organizer, \href{https://etap4.krisyu.org}{Experimental and Theoretical Advances in
        Prosody 4}. October 11-13, 2018. With Meghan Armstrong-Abrami, Mara Breen, and Heather Pon-Barry.
      
    \item Co-organizer, \href{https://etap4.krisyu.org/aae}{Experimental and Theoretical Advances in
        Prosody 4: Satellite Workshop on African American Language Prosody}. October 10, 2018. With Meghan Armstrong-Abrami, Mara Breen, and Heather Pon-Barry.
      
        \item Organizer, UMass Tone Workshop (June 2-3,
          2014). International workshop.
        \begin{innerlist}[\enskip$\circ$,leftmargin=*]
            \item Principle Investigator, Andrew Mellon Foundation / Center for Teaching \& Faculty
      Development at the University of Massachusetts Amherst, ``Five
      Colleges Prosody Community'', \$10,000, 2013 -- 2014.
        \end{innerlist}

        \halfblankline

        \item Organizing committee, 5th Tonal Aspects of Language (TAL)
          Symposium, 2015 (May 19-22, 2016)

        \item Scientific committee, Speech Prosody 2016 (May 31-June
          3, 2016)

        \end{innerlist}

\blankline

\textbf{Editing}

\begin{innerlist}

\item Editorial Board, \textit{Journal of South Asian Linguistics},
  2016 --
\item Editorial Board, \textit{Journal of the International Phonetic
    Association}, 2016 --  
\item Co-editor, Special Issue of \textit{Laboratory
    Phonology} on prosodic variability, 2015 -- 2016
\end{innerlist}

\blankline

\textbf{Ad-hoc reviewing}

\begin{outerlist}

    \item[] \textit{Journals}%

      \begin{innerlist}
      \item
        \href{http://www.journal-labphon.org/}{\emph{Laboratory Phonology}}, 2017
      \item
        \href{http://jsal-journal.org/}{\emph{Journal of South Asian Linguistics}}, 2017
      \item
        \href{http://www.plosone.org/}{\emph{PLoS ONE}}, 2015
      \item
        \href{http://las.sagepub.com}{\emph{Language
            and Speech}}, 2015 -- present
      \item
        \href{http://www.springer.com/psychology/cognitive+psychology/journal/13414}{\emph{Attention,
            Perception, and Psychophysics}}, 2015
      \item
        \href{http://link.springer.com/journal/10831}{\emph{Journal of East Asian Linguistics}},
        2015
      \item
        \href{http://journals.cambridge.org/action/displayJournal?jid=PHO}{\emph{Phonology}},
        2014 -- 2015
      \item \href{http://www.journals.elsevier.com/lingua/}{\emph{Lingua}}, 2013 -- 2015
        \item
          \href{http://journals.cambridge.org/action/displayJournal?jid=IPA}{\emph{Journal
              of the International Phonetic Association}}, 2013 --- present
        \item \href{http://nflrc.hawaii.edu/ldc/}{\emph{Language Documentation \& Conservation}}, 2013
        \item \href{http://asadl.org/jasa/}{\emph{Journal of the Acoustical Society of America}}, 2011 -- 2013
        \item \href{http://www.journals.elsevier.com/journal-of-phonetics/}{\emph{Journal of Phonetics}}, 2011
      \end{innerlist}

    \item[] \textit{Grants}%

      \begin{innerlist}
      \item \emph{National Science Foundation Documenting Endangered Languages Program}, 2017 - present
      \item \emph{National Science Foundation Linguistics Program}, 2014
      \end{innerlist}

    \item[] \textit{Conferences}%

      \begin{innerlist}
      \item \emph{Generative Linguistics in the Old World, Asia}, 2016
      \item \emph{International Congress of Phonetic Sciences}, 2015, 2019
      \item \emph{Austronesian Formal Linguistics Association}, 2015
        -- present
      \item \emph{Generative Linguistics in the Old World}, 2015 -- present
      \item \emph{ACL: SIGMORPHON}, 2014 -- present
      \item \emph{Exploring the Interfaces 3}, 2014
        \item \emph{Tonal Aspects of Language}, 2014
        \item \emph{Tone and Intonation in Europe}, 2016
        \item \emph{Annual Meetings on Phonology}, 2013 -- present
        \item \emph{Linguistic Society of America Annual Meeting},
          2012 -- 2014, 2018 -- present
        \item \emph{West Coast Conference on Formal Linguistics}, 2012 -- 2014
      \end{innerlist}

\end{outerlist}

\section{Professional Memberships}

\begin{innerlist}
\item Linguistic Society of America (2007 -- present)
\item Acoustical Society of America (2008 -- present)
\item International Speech Communication Association (2008 -- present)
\item Cognitive Science Society (2010 -- 2012)
\item Association for Laboratory Phonology (2015 -- present)
\end{innerlist}

\section{Departmental service}

\begin{innerlist}
\item Undergraduate advisor, 2017 -- present
\item Strategic Planning for Undergraduate Education: research sub-committee, 2015
\item Search Committee, 2014 -- 2015
\item Annual departmental professional development workshop series for
  graduate students: giving talks and posters, 2014 -- present
  \item Library liaison, 2013 -- 2016
  \item Admissions Committee, 2012 -- 2013
  \item Personnel Committee, 2011 -- present
\end{innerlist}

\section{University service}
\begin{innerlist}
\item College of Humanities and Fine Art Curriculum Committee, 2017 -- 2019
\item Graduate School Grant Committee Member, 2017 -- present
\item Graduate Council Member, 2017 -- 2019
\item \textit{Faculty Voices} series interview (original interview footage of
  UMass faculty who are engaging in diverse and innovative teaching
  practices), on the theme of collaborative learning. Institute of Teaching Excellence \& Faculty
  Development. 2016 -- 2017.
\item UMass Graduate School Office of Professional Development Workshop on National Science Foundation Doctoral Dissertation Research Improvements Grants, faculty panel member. June 19, 2017.
\end{innerlist}

\section{Five College Consortium service}
\begin{innerlist}
\item Five Colleges Prosody Group co-founder and co-director, 2013 -- present.
\item Linguistics at the Five Colleges Facebook group co-founder/administrator, 2016 -- present [\href{https://www.facebook.com/groups/ling5}{webpage}].
\end{innerlist}


\section{Outreach}

\begin{innerlist}
\item The Science of Language, linguistics demonstrations at Jackson
  Street School Family Science Night. November 17, 2016.
  \item Talk on tone and intonation at MacDuffie School Linguistics
    after-school program, February 27, 2014.
  \item Talk on prosody at Williston-Northampton School linguistics
    high school class, March 26, 2015.
\end{innerlist}

\end{document}

%%%%%%%%%%%%%%%%%%%%%%%%%% End CV Document %%%%%%%%%%%%%%%%%%%%%%%%%%%%%

%----------------------------------------------------------------------%
% The following is copyright and licensing information for
% redistribution of this LaTeX source code; it also includes a liability
% statement. If this source code is not being redistributed to others,
% it may be omitted. It has no effect on the function of the above code.
%----------------------------------------------------------------------%
% Copyright (c) 2007, 2008, 2009, 2010, 2011 by Theodore P. Pavlic
%
% Unless otherwise expressly stated, this work is licensed under the
% Creative Commons Attribution-Noncommercial 3.0 United States License. To
% view a copy of this license, visit
% http://creativecommons.org/licenses/by-nc/3.0/us/ or send a letter to
% Creative Commons, 171 Second Street, Suite 300, San Francisco,
% California, 94105, USA.
%
% THE SOFTWARE IS PROVIDED "AS IS", WITHOUT WARRANTY OF ANY KIND, EXPRESS
% OR IMPLIED, INCLUDING BUT NOT LIMITED TO THE WARRANTIES OF
% MERCHANTABILITY, FITNESS FOR A PARTICULAR PURPOSE AND NONINFRINGEMENT.
% IN NO EVENT SHALL THE AUTHORS OR COPYRIGHT HOLDERS BE LIABLE FOR ANY
% CLAIM, DAMAGES OR OTHER LIABILITY, WHETHER IN AN ACTION OF CONTRACT,
% TORT OR OTHERWISE, ARISING FROM, OUT OF OR IN CONNECTION WITH THE
% SOFTWARE OR THE USE OR OTHER DEALINGS IN THE SOFTWARE.
%----------------------------------------------------------------------%
